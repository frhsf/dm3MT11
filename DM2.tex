\documentclass[12pt,a4paper, french]{article} 
\usepackage[T1]{fontenc}              
\usepackage[utf8]{inputenc}  
\usepackage[french]{babel}
\frenchbsetup{StandardLists=true}
\usepackage{graphicx}
\usepackage{amssymb}
\usepackage{amsmath}
\usepackage[hidelinks]{hyperref}
\usepackage[left=2cm,right=2cm,top=2cm,bottom=2cm]{geometry}
\usepackage[skip=0.13cm]{parskip}
\usepackage{times}
\usepackage{amsmath}
\newcommand\e{\mathrm{e}}
\setlength\parindent{20pt}

\title{DM3 MT11}
\author{Léo Eugène, Mathieu Poveda, Sacha Hénaff}
\date{}
\begin{document}
\maketitle

\section*{Exercice 1}
On cherche:
\begin{equation*}
    \int \frac{x}{\sqrt{1+x^2}}\,dx
\end{equation*}

On pose \begin{math}
    u=1+x^2
\end{math} donc \begin{math}
    \,du = 2x\,dx 
\end{math} et ainsi: \begin{math}
    \,dx=\frac{1}{2x}du
\end{math}. On obtient donc: 

\begin{equation*}
    \int \frac{x}{\sqrt{1+x^2}}\,dx=\frac{1}{2}\int \frac{1}{\sqrt{u}}\,du=\frac{1}{2}2\sqrt{u}+C=\sqrt{u}+C=\sqrt{1+x^2}+C
\end{equation*}

 On a:
 \begin{equation*}
    \frac{1}{1+x^2}
 \end{equation*}

On a donc:
\begin{center}
    \begin{equation*}
        \int \frac{1}{1+x^2}\,dx = \arctan(x)+C
    \end{equation*}
\end{center}

De même, on a: 
\begin{equation*}
    \frac{x}{1+x^2}
 \end{equation*}

On remarque que cette fonction est de la forme \begin{math}
    \frac{u'}{u}
\end{math} en mettant \begin{math}
    \frac{1}{2}
\end{math} en facteur. On a donc:
\begin{equation*}
    \int \frac{x}{1+x^2} \,dx= \frac{1}{2}\ln(x^2+1)
 \end{equation*}

 Dans ce cas, il n'est pas nécessaire de mettre une valeur absolue car \begin{math}
    x^2+1>0
 \end{math}

 On cherche:
\begin{equation*}
    \int e^{6x}(x^2+3x+1)
\end{equation*}

En utilisant l'intégration par partie avec \begin{math}
    u'=e^{6x}
\end{math} donc \begin{math}
    u=\frac{1}{6}e^{6x}
\end{math} et \begin{math}
    v=x^2+3x+1
\end{math} et donc \begin{math}
    v'=2x+3
\end{math}, on obtient:

\begin{equation*}
    \int e^{6x}(x^2+3x+1)=\left[\frac{1}{6}e^{6x}(x^2+3x+1)\right]-\int \frac{1}{6}(2x+3)e^{6x}\,dx
\end{equation*}
\newpage

Donc: 
\begin{equation*}
    \int e^{6x}(x^2+3x+1)=\left[\frac{1}{6}e^{6x}(x^2+3x+1)\right]-\int \frac{e^{6x}}{2}\,dx-\int \frac{x}{3}e^{6x}\,dx 
\end{equation*}

Et donc:
\begin{equation*}
    \int e^{6x}(x^2+3x+1)=\left[\frac{1}{6}e^{6x}(x^2+3x+1)\right]-\left[\frac{e^{6x}}{12}\right]-\int \frac{x}{3}e^{6x}\,dx
\end{equation*}

On doit faire une deuxième intégration par partie avec:\begin{math}
    u'=e^{6x}
\end{math} donc \begin{math}
    u=\frac{1}{6}e^{6x}
\end{math} et \begin{math}
    v=\frac{x}{3}
\end{math} et donc \begin{math}
    v'=\frac{1}{3}
\end{math}. On obtient donc:
\begin{equation*}
    \int e^{6x}(x^2+3x+1)=\left[\frac{1}{6}e^{6x}(x^2+3x+1)\right]-\left[\frac{e^{6x}}{12}\right]-\left[e^{6x}\frac{x}{18}\right]+\int \frac{e^6x}{18}\,dx
\end{equation*}

Donc:
\begin{equation*}
    \int e^{6x}(x^2+3x+1)=\left[\frac{1}{6}e^{6x}(x^2+3x+1)\right]-\left[\frac{e^{6x}}{12}\right]-\left[e^{6x}\frac{x}{18}\right]+\left[\frac{e^{6x}}{108}\right]+C
\end{equation*}

En factorisant par \begin{math}
    \frac{e^{6x}}{6}
\end{math}, on obtient:

\begin{equation*}
    \int e^{6x}(x^2+3x+1)=\frac{e^{6x}}{6}\left(x^2+3x+1-\frac{1}{2}-\frac{x}{3}+\frac{1}{18}\right)+C
\end{equation*}

Ce qui nous donne finalement:

\begin{equation*}
    \int e^{6x}(x^2+3x+1)=\frac{e^{6x}}{6}\left(x^2+\frac{8x}{3}+\frac{10}{18}\right)+C
\end{equation*}
 \section*{Exercice 2}
 1) On a: \begin{equation*}
    \frac{x+2}{x+1}=a+\frac{b}{x+1}
 \end{equation*}

 On cherche a et b qui vérifient cette égalité. On a donc:

\begin{equation*}
    a+\frac{b}{x+1}=\frac{a(x+1)+b}{x+1}=\frac{ax+a+b}{x+1}
\end{equation*}

On a donc par identification: \begin{math}
    ax=x \Leftrightarrow a= 1 
\end{math} et donc: \begin{math}
    a+b=2 \Leftrightarrow b=1
\end{math} On a donc: 

\begin{equation*}
    \frac{x+2}{x+1}=1+\frac{1}{x+1}
\end{equation*}

Ainsi: 
\begin{equation*}
    \int \frac{x+2}{x+1} \,dx= \int \left(1+\frac{1}{x+1}\right)\,dx=\int 1\,dx+\int \frac{1}{x+1} \,dx=x+\ln(\left\lvert x+1\right\rvert )+C
\end{equation*}

2) On a:
\begin{equation*}
    \frac{x+2}{3x^2+x-2}=\frac{a}{x+1}+\frac{b}{x-\frac{2}{3}}=\frac{a(x-\frac{2}{3})+b(x-1)}{(x+1)(x-\frac{2}{3})}
\end{equation*}

En développant le dénominateur, on a:
\begin{equation*}
    x^2-\frac{2}{3}x+x-\frac{2}{3}=x^2+\frac{1}{3}x-\frac{2}{3}
\end{equation*}

On voit qu'il s'agit d'un tiers du dénominateur de la fonction de départ. Or pourpouvoir comparer les deux numératuers, il faut que les dénominateurs soient égaux.  On a donc:
\begin{equation*}
    \frac{x+2}{3x^2+x-2}=\frac{x+2}{3(x+1)(x-\frac{2}{3})}
\end{equation*}


Et donc: 
\begin{equation*}
    \frac{x+2}{3(x+1)(x-\frac{2}{3})}=\frac{a(x-\frac{2}{3})+b(x-1)}{(x+1)(x-\frac{2}{3})}
\end{equation*}

On a donc: \begin{math}
    \frac{x+2}{3}=ax-\frac{2a}{3}+b
\end{math}
  
Et en identifiant les termes en x et les autres on obtient: 
\newline
\begin{center}
$\left\{
    \begin{array}{ll}
        ax+bx=\frac{x}{3}\\
        \frac{-2a}{3}+b=\frac{2}{3}\\ 
    \end{array}
\right.$
\end{center}

On obtient donc: \begin{math}
a=-\frac{1}{5}
\end{math} et \begin{math}
    b=\frac{8}{15}
\end{math}

Ainsi, on a: 
\begin{equation*}
    \frac{x+2}{3x^2+x-2}=\frac{\frac{-1}{5}}{x+1}+\frac{\frac{8}{15}}{x-\frac{2}{3}}
\end{equation*}

Donc: 
\begin{equation*}
\int \frac{x+2}{3x^2+x-2} \,dx = \frac{-1}{5}\int \frac{1}{x+1}\,dx+\frac{8}{15} \int \frac{1}{x-\frac{2}{3}} \,dx=-\frac{1}{5}\ln\left\lvert x+1 \right\rvert +\frac{8}{15}\ln\lvert x-\frac{2}{3}\rvert +C
\end{equation*}
3) a. On a: 
\begin{equation*}
    \int \frac{1}{(x-a)^2+b^2} \,dx=\int \frac{1}{b^2}\frac{1}{(\frac{x-a}{b})^2+1} \,dx
\end{equation*}
 
On pose: \begin{math}
t=\frac{x-a}{b}    
\end{math} donc \begin{math}
    \,dt=\frac{1}{b}\,dx
\end{math} et donc \begin{math}
    \,dx=b\,dt
\end{math}

On obtient donc: 
\begin{equation*}
    \int \left( \frac{1}{b^2}\frac{1}{(\frac{x-a}{b})^2+1} \right)\,dx=\int \left(\frac{1}{b^2} \frac{b}{t^2+1}\right)\,dt=\frac{1}{b}\int \frac{1}{t^2+1}\,dt= \frac{1}{b}\arctan(t)+C
\end{equation*}

Ainsi, on a en remplaçant t par son expression: 
\begin{equation*}
    \int \frac{1}{(x-a)^2+b^2} \,dx=\frac{1}{b}\arctan \left(\frac{x-a}{b}\right)+C
\end{equation*}

b. On cherche à calculer: 
\begin{equation*}
    F_1(x)=\int \frac{1}{x^2-2x+3}\,dx 
\end{equation*}

En remarquand que le dénominateur est le début du dévloppement d'une identité ramarquable, on a:
\begin{equation*}
    \int \frac{1}{x^2-2x+3}\,dx = \int \frac{1}{(x-1)^1+2}\,dx
\end{equation*}

En identifiant \begin{math}
    a=1
\end{math} et \begin{math}
    b=\sqrt{2}
\end{math}. On obtient:
\begin{equation*}
    F_1(x)=\int \frac{1}{x^2-2x+3}\,dx=\frac{1}{\sqrt{2}}\arctan\left(\frac{x-1}{\sqrt{2}}\right)+C
\end{equation*}

c. On cherche à calculer: 

\begin{equation*}
    F_2(x)=\int \frac{2x-2}{x^2-2x+3}\,dx
\end{equation*}

On voit que cette intégrale est de la forme \begin{math}
    \frac{u'}{u}
\end{math}. Donc:
\begin{equation*}
    F_2(x)=\int \frac{2x-2}{x^2-2x+3}\,dx=\ln\lvert x^2-2x+3\rvert +C
\end{equation*}

d. On cherche à calculer:

\begin{equation*}
    F_3(x)=\int \frac{3x+5}{x^2-2x+3}\,dx
\end{equation*}

On cherche a et b tels que \begin{math}
    a(2x-2)+b=3x+5
\end{math}. On obtient \begin{math}
    a=\frac{3}{2}
\end{math} et \begin{math}
    b=8
\end{math}. Donc:
\begin{equation*}
    F_3(x)=\int \frac{3x+5}{x^2-2x+3}\,dx=\frac{3}{2}F_2(x)+8F_1(x)=\frac{3}{2}\ln\lvert x^2-2x+3\rvert+\frac{8}{\sqrt{2}}\arctan\left(\frac{x-1}{\sqrt{2}}\right)+C
\end{equation*}
\end{document}